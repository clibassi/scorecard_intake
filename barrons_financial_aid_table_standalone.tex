\documentclass[12pt]{article}
\usepackage[margin=1in]{geometry}
\usepackage{booktabs}
\usepackage{caption}
\usepackage{siunitx}
\usepackage{hyperref}
\usepackage{tabularx}

\sisetup{group-separator={,}, group-minimum-digits=4}

\begin{document}

\begin{table}[p]
\centering
\caption{Financial Aid and Net Price by College Selectivity Category}
\label{tab:financial_aid}
\footnotesize
\begin{tabular}{@{}l r r r r r r r r r r@{}}
\toprule
 & & & \multicolumn{2}{c}{Aid Rates} & \multicolumn{6}{c}{Median Net Price by Family Income} \\
\cmidrule(lr){4-5} \cmidrule(lr){6-11}
Category & N & Undergrads & Pell & Loan & Overall & \$0--30k & \$30--48k & \$48--75k & \$75--110k & \$110k+ \\
\midrule
Most competitive & 89 & 518,402 & 15.6 & 21.4 & 28,466 & 8,865 & 8,640 & 14,569 & 22,768 & 49,113 \\
HC+ & 37 & 338,710 & 19.4 & 38.4 & 24,954 & 15,399 & 15,828 & 19,150 & 25,274 & 34,644 \\
HC & 68 & 677,746 & 20.9 & 39.1 & 25,810 & 15,885 & 15,813 & 19,998 & 26,084 & 33,832 \\
VC+ & 55 & 406,952 & 21.7 & 42.3 & 24,546 & 17,213 & 17,633 & 21,634 & 25,618 & 30,372 \\
VC & 213 & 1,493,809 & 26.3 & 49.2 & 21,782 & 16,002 & 16,532 & 19,850 & 24,263 & 27,449 \\
C+ & 70 & 343,767 & 34.1 & 53.4 & 20,482 & 16,532 & 15,933 & 18,504 & 22,378 & 24,844 \\
C & 574 & 2,894,014 & 33.8 & 53.2 & 19,602 & 14,906 & 15,728 & 18,494 & 21,803 & 23,955 \\
LC & 151 & 581,098 & 41.3 & 50.8 & 15,951 & 13,318 & 13,754 & 16,327 & 19,263 & 20,766 \\
NC & 62 & 245,181 & 39.4 & 38.5 & 14,152 & 11,436 & 12,166 & 14,676 & 17,982 & 18,996 \\
Specialized programs & 65 & 168,078 & 28.4 & 47.5 & 34,725 & 27,585 & 28,423 & 33,385 & 36,080 & 38,839 \\
No category & 4,765 & 6,406,776 & 44.5 & 40.9 & 15,071 & 13,965 & 13,624 & 14,941 & 16,925 & 19,174 \\
\bottomrule
\end{tabular}


\vspace{1em}
\begin{minipage}{\textwidth}
\footnotesize

\textit{Notes:} This table presents median values across institutions within each Barron's selectivity category. 
N indicates the number of institutions in each category. Undergrads shows total undergraduate enrollment summed across all institutions. 
Aid rates are expressed as percentages. Net price represents the average annual cost of attendance after all grant aid 
(federal, state, and institutional), reported separately by family income level. All dollar amounts are nominal (not adjusted for inflation).

\vspace{0.5em}

\textit{Data Sources:} College selectivity categories from Barron's Educational Series, Inc., \textit{Barron's Profiles of American Colleges}, 
29th edition (Hauppauge, NY: Barron's Educational Series, 2013). 
Financial aid, net price, and enrollment data from U.S.\ Department of Education, College Scorecard, 
accessed at \url{https://collegescorecard.ed.gov/data}. 
Analysis sample includes 6,149 undergraduate-degree-granting institutions (excludes 280 graduate-only institutions).

\vspace{0.5em}

\textit{Variable Definitions and Measurement Periods:} 

\textbf{Pell Grant Rate} (PCTPELL): The proportion of all undergraduates who received a Pell grant in the academic year, 
calculated as the number of Pell grant recipients divided by total undergraduate enrollment. 
Data from Integrated Postsecondary Education Data System (IPEDS) Student Financial Aid component, Academic Year 2022--23, 
reported in IPEDS Data Collection Year 2023--24.

\textbf{Federal Loan Rate} (PCTFLOAN): The proportion of all undergraduates who received a federal student loan in the academic year. 
Data from IPEDS Student Financial Aid component, Academic Year 2022--23, reported in IPEDS DCY 2023--24.

\textbf{Overall Net Price} (NPT4\_PUB for public institutions, NPT4\_PRIV for private institutions): 
The average annual total cost of attendance (including tuition and fees, books and supplies, and living expenses) 
minus the average grant/scholarship aid, calculated for all full-time, first-time, degree/certificate-seeking undergraduates who receive Title IV aid. 
For public institutions, this metric is limited to students paying in-state tuition. 
For private institutions, it includes all full-time, first-time, degree/certificate-seeking undergraduates receiving Title IV aid. 
Net price represents an average across all programs for academic-year institutions, including only students who first enrolled in fall term. 
The total cost of attendance varies by whether students live on campus, off campus (not with family), or off campus (with family). 
Data from IPEDS Student Financial Aid component, Academic Year 2022--23, reported in IPEDS DCY 2023--24.

\textbf{Net Price by Family Income} (NPT41--45\_PUB for public, NPT41--45\_PRIV for private): 
Average net price calculated separately for five family income categories: \$0--30,000 (NPT41); \$30,001--48,000 (NPT42); 
\$48,001--75,000 (NPT43); \$75,001--110,000 (NPT44); and \$110,001 or more (NPT45). 
Income categories are based on the income values used by the institution to calculate financial aid eligibility. 
For dependent students, this includes the student's and the parents' income. 
For independent students, this includes the student's adjusted gross income and, if married, the spouse's income. 
Same measurement criteria as overall net price. 
Data from IPEDS Student Financial Aid component, Academic Year 2022--23, reported in IPEDS DCY 2023--24.

\textbf{Undergraduate Enrollment} (UGDS): Total enrollment of undergraduate certificate/degree-seeking students. 
Data from IPEDS Fall Enrollment component, Fall 2023, reported in IPEDS DCY 2023--24.

\vspace{0.5em}

\textit{Barron's Selectivity Categories:} Most competitive through NC (Noncompetitive) represent Barron's nine selectivity ratings based on 
median entrance examination scores, percentages of students scoring 500 and above and 600 and above on SAT sections (or equivalent ACT scores), 
percentage of applicants accepted, percentage of freshmen from top tenth and top quarter of high school class, and minimum class rank and 
grade point average required for admission. 
``Specialized programs'' includes arts conservatories, military academies, and specialized professional schools that Barron's rates separately 
from the main selectivity scale. 
``No category'' includes 4,765 institutions (78.5\% of sample) not included in Barron's 2013 ratings. 
These primarily comprise community colleges, vocational and technical schools, certificate-granting institutions, and other non-selective 
post-secondary institutions.

\end{minipage}

\end{table}

\end{document}