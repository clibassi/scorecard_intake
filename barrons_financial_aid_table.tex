\begin{table}[htbp]
\centering
\caption{Financial Aid and Net Price by College Selectivity Category}
\label{tab:financial_aid}
\footnotesize
\begin{tabular}{@{}l r r r r r r r r r r@{}}
\toprule
 & & & \multicolumn{2}{c}{Aid Rates} & \multicolumn{6}{c}{Net Price by Family Income} \\
\cmidrule(lr){4-5} \cmidrule(lr){6-11}
Category & N & Undergrads & Pell & Loan & Overall & \$0--30k & \$30--48k & \$48--75k & \$75--110k & \$110k+ \\
\midrule
Most competitive & 89 & 518,402 & 17.7 & 22.0 & 26,803 & 9,245 & 9,885 & 13,620 & 22,928 & 46,378 \\
HC+ & 37 & 338,710 & 17.5 & 27.2 & 21,239 & 10,147 & 10,974 & 15,388 & 22,417 & 30,713 \\
HC & 68 & 677,746 & 21.8 & 31.6 & 22,165 & 12,721 & 13,638 & 17,697 & 24,548 & 32,015 \\
VC+ & 55 & 406,952 & 22.2 & 35.5 & 23,211 & 15,620 & 17,207 & 21,186 & 25,666 & 29,505 \\
VC & 213 & 1,493,809 & 27.0 & 38.8 & 19,993 & 13,436 & 14,557 & 17,959 & 22,788 & 26,517 \\
C+ & 70 & 343,767 & 35.9 & 38.8 & 15,193 & 11,110 & 11,824 & 14,696 & 18,515 & 21,383 \\
C & 574 & 2,894,014 & 35.7 & 43.5 & 17,923 & 14,003 & 14,771 & 17,444 & 21,127 & 23,530 \\
LC & 151 & 581,098 & 42.7 & 41.2 & 13,953 & 11,091 & 11,883 & 14,491 & 17,924 & 20,236 \\
NC & 62 & 245,181 & 37.9 & 31.7 & 13,758 & 11,404 & 12,114 & 14,474 & 17,547 & 19,412 \\
Specialized programs & 65 & 168,078 & 27.4 & 32.9 & 28,519 & 25,287 & 31,670 & 34,808 & 39,505 & 41,844 \\
No category & 4,765 & 6,406,776 & 34.4 & 20.2 & 10,139 & 8,777 & 9,241 & 11,209 & 13,488 & 14,681 \\
\bottomrule
\end{tabular}


\vspace{1em}
\begin{minipage}{\textwidth}
\footnotesize

\textit{Notes:} Table presents enrollment-weighted means within each Barron's selectivity category, where larger institutions receive proportionally greater weight. 
Pell and loan rates are percentages of all degree/certificate-seeking undergraduates receiving Pell grants and federal loans, respectively (PCTPELL and PCTFLOAN variables). 
Net price represents average annual cost of attendance minus all grant aid for full-time, first-time, degree/certificate-seeking undergraduates receiving Title IV aid; 
for public institutions, limited to in-state tuition. 
Income-specific net price calculated separately for five family income brackets using income values for financial aid eligibility. 
All dollar amounts are nominal. 
College selectivity categories from Barron's Educational Series, \textit{Barron's Profiles of American Colleges}, 30th edition (Hauppauge, NY: Barron's Educational Series, 2012). 
Financial aid and enrollment data from U.S.\ Department of Education, \textit{College Scorecard Data}, April 2025 release, Office of Planning, Evaluation and Policy Development (\url{https://collegescorecard.ed.gov/data/}), 
Academic Year 2022--23 for Pell grant rates, federal loan rates, and net price (IPEDS Student Financial Aid component), 
Fall 2023 for undergraduate enrollment (IPEDS Fall Enrollment component). 
Sample includes 6,149 undergraduate-degree-granting institutions. 
``No category'' includes 4,765 institutions (78.5\%) not rated by Barron's, primarily community colleges and vocational schools.

\end{minipage}

\end{table}